\documentclass[11pt]{article}
\usepackage{amsmath,amsfonts,amssymb,amsthm}
\usepackage{graphicx, tabularx, algorithmic, algorithm}
\usepackage{cite}
\usepackage{mathtools}
\usepackage{subfig}


\title{\textbf{Spiking Neural Networks for Remaining Useful Life Prediction:\\
An Ablation Study on Spike Encoding Methods using the CMAPSS Dataset}}

\author{
Vishakh Hari\\
\textit{Email: vishakh.hari.08@gmail.com}
}

\date{\today}

\begin{document}

\maketitle

\begin{abstract}
This paper presents a comprehensive implementation of Spiking Neural Networks (SNNs) for Remaining Useful Life (RUL) prediction of turbofan engines using NASA's Commercial Modular Aero-Propulsion System Simulation (CMAPSS) dataset. We implement a neuromorphic computing approach that leverages Leaky Integrate-and-Fire (LIF) neurons with surrogate gradient training to predict engine failure timelines. Our study conducts an ablation analysis comparing rate encoding versus temporal encoding methods for converting continuous sensor data into spike trains. The SNN architecture incorporates recurrent dynamics, membrane potential evolution, and spike-based information processing to capture temporal dependencies in engine degradation patterns. Experimental results demonstrate the effectiveness of spike-based neural computation for prognostics applications, with particular emphasis on the energy efficiency and biological plausibility advantages of neuromorphic approaches. The implementation utilizes the snnTorch framework with PyTorch backend, achieving competitive performance while maintaining the inherent temporal processing capabilities of spiking neural networks. This work contributes to the growing field of neuromorphic computing applications in aerospace industrial prognostics and health management systems.
\end{abstract}

\section{Introduction}

Remaining Useful Life (RUL) prediction is a critical component of predictive maintenance strategies in aerospace and industrial applications. Traditional approaches using conventional neural networks have achieved significant success, but they often require substantial computational resources and lack the temporal dynamics inherent in biological neural processing \cite{saxena2008damage}. Spiking Neural Networks (SNNs) offer a neuromorphic computing paradigm that naturally processes temporal information through discrete spike events, making them well-suited for time-series prognostics applications.

The motivation for applying SNNs to RUL prediction stems from several key advantages: (1) inherent temporal processing capabilities that align with the sequential nature of sensor data, (2) energy-efficient computation through sparse spike-based activation, and (3) the biological plausibility of neural dynamics that may capture complex degradation patterns more effectively than traditional approaches.

This paper makes the following contributions: (1) A complete implementation of SNNs for turbofan engine RUL prediction using the CMAPSS dataset, (2) an ablation study comparing rate encoding and temporal encoding methods for spike train generation, (3) a comprehensive evaluation framework incorporating both traditional regression metrics and prognostics-specific scoring functions, and (4) detailed analysis of spiking neural dynamics in the context of industrial time-series forecasting.

The remainder of this paper is organized as follows: Section 2 provides background on spiking neural networks and neuromorphic computing principles, Section 3 describes the CMAPSS dataset characteristics and preprocessing pipeline, Section 4 details our SNN implementation and methodology, Section 5 presents experimental results and analysis, and Section 6 concludes with discussion and future directions.

\section{Dataset Description: NASA CMAPSS}

The Commercial Modular Aero-Propulsion System Simulation (CMAPSS) dataset represents a cornerstone benchmark for RUL prediction research in aerospace applications. Developed by NASA, this dataset provides realistic turbofan engine degradation data through high-fidelity simulation of commercial aircraft engines under various operating conditions and fault scenarios.

\subsection{Dataset Composition and Variants}

The CMAPSS dataset comprises four distinct sub-datasets (FD001-FD004), each designed to evaluate different aspects of prognostics algorithm performance. Table \ref{tab:cmapss_variants} summarizes the key characteristics of each variant.

\begin{table}[h]
\centering
\caption{CMAPSS Dataset Variants}
\label{tab:cmapss_variants}
\begin{tabular}{lcccc}
\textbf{Dataset} & \textbf{Train} & \textbf{Test} & \textbf{Conditions} & \textbf{Faults} \\
FD001 & 100 & 100 & 1 & 1 \\
FD002 & 260 & 259 & 6 & 1 \\
FD003 & 100 & 100 & 1 & 2 \\
FD004 & 248 & 249 & 6 & 2 \\
\end{tabular}
\end{table}

Each dataset variant introduces increasing complexity: FD001 serves as a baseline with single operating condition and fault mode, while FD004 represents the most challenging scenario with multiple operating conditions and fault modes simultaneously. This progression allows for systematic evaluation of algorithm robustness across different operational complexities.

\subsection{Data Structure and Sensor Configuration}

Each CMAPSS data point consists of 26 columns: engine unit identifier, time cycle, three operational settings, and 21 sensor measurements. The operational settings represent flight altitude, Mach number, and throttle resolver angle, capturing the operational envelope of commercial flight operations. The 21 sensors monitor various engine parameters including temperatures, pressures, speeds, and fuel flow rates across different engine components.

The multivariate time series structure captures engine behavior from healthy operation through progressive degradation to failure. Training trajectories span the complete engine lifecycle, while test trajectories are truncated at random points before failure, requiring algorithms to predict the remaining operational cycles.

\subsection{Fault Modes and Degradation Patterns}

The dataset incorporates two primary fault modes affecting critical engine components:

\textbf{High Pressure Compressor (HPC) Degradation:} This fault mode simulates efficiency reduction in the high-pressure compressor stage, affecting overall engine performance and fuel consumption patterns.

\textbf{Fan Degradation:} This fault affects the engine fan assembly, impacting airflow characteristics and pressure ratios throughout the engine.

These fault modes can occur individually (FD001, FD002) or simultaneously (FD003, FD004), creating realistic degradation scenarios that mirror actual engine failure patterns observed in operational environments.

\subsection{Preprocessing and Normalization}

Our preprocessing pipeline implements several key steps to prepare CMAPSS data for SNN training:

\textbf{Normalization:} Individual sensor channels are normalized using min-max scaling to ensure consistent input ranges for spike encoding. Engine-specific normalization is applied to account for manufacturing variations and initial wear states.

\textbf{Sequence Windowing:} Time series data is segmented into fixed-length sequences to provide temporal context for RUL prediction. Window length selection balances computational efficiency with sufficient temporal information for degradation pattern recognition.

\textbf{RUL Calculation:} For training data, RUL values are computed as the difference between maximum engine lifetime and current cycle number. A piecewise linear approach is implemented to handle early-life periods where degradation patterns may not be evident.

\section{Spiking Neural Networks and Neuromorphic Computing}

Spiking Neural Networks represent the third generation of artificial neural networks, distinguished by their use of discrete spike events for information transmission and processing. Unlike conventional neural networks that employ continuous activation functions, SNNs communicate through binary spike trains with precise temporal encoding, closely mimicking biological neural computation.

\subsection{Biological Motivation and Computational Principles}

The fundamental motivation for SNNs stems from neuroscience research demonstrating that biological neurons communicate through action potentials—brief electrical impulses that propagate through neural networks. This spike-based communication enables the brain to process information with remarkable energy efficiency, consuming approximately 20 watts while performing complex cognitive tasks that challenge modern supercomputers.

SNNs incorporate several key biological principles: (1) \textbf{Temporal sparsity}, where neurons fire infrequently, leading to energy-efficient computation, (2) \textbf{Membrane dynamics}, where neurons accumulate input signals over time before generating output spikes, and (3) \textbf{Synaptic plasticity}, enabling learning through spike-timing-dependent mechanisms.

\subsection{Leaky Integrate-and-Fire Neuron Model}

The Leaky Integrate-and-Fire (LIF) neuron serves as the fundamental computational unit in our SNN implementation. The LIF model captures essential neural dynamics through a differential equation governing membrane potential evolution:

\begin{equation}
\tau_m \frac{du}{dt} = -(u - u_{rest}) + I(t)
\end{equation}

where $u$ represents membrane potential, $\tau_m$ is the membrane time constant, $u_{rest}$ is the resting potential, and $I(t)$ is the input current. When the membrane potential exceeds a threshold $u_{th}$, the neuron generates a spike and resets to the resting potential.

The discrete-time implementation used in our system follows:

\begin{equation}
u[t] = \beta u[t-1] + I[t]
\end{equation}

where $\beta = \exp(-\Delta t / \tau_m)$ represents the decay factor, incorporating the leaky integration property that enables temporal information processing.

\subsection{Surrogate Gradient Training}

A fundamental challenge in SNN training is the non-differentiable nature of spike generation, which prevents standard backpropagation. The spike function can be represented as:

\begin{equation}
S = \Theta(u - u_{th})
\end{equation}

where $\Theta$ is the Heaviside step function. The derivative of this function is zero everywhere except at the threshold, where it is undefined, making gradient-based optimization impossible.

Surrogate gradient methods address this challenge by replacing the spike function derivative during backpropagation while maintaining discrete spikes during forward propagation. Our implementation employs the fast sigmoid surrogate gradient:

\begin{equation}
\frac{dS}{du} = \frac{1}{1 + (s \cdot |u - u_{th}|)^2}
\end{equation}

where $s$ controls the surrogate gradient sharpness. This approach enables end-to-end training of deep SNNs while preserving spike-based computation.

\subsection{Neuromorphic Hardware and Energy Efficiency}

SNNs are naturally suited for neuromorphic hardware implementations that promise significant energy efficiency improvements over conventional von Neumann architectures. Neuromorphic chips such as Intel's Loihi and IBM's TrueNorth exploit temporal sparsity in spike trains, consuming power only during spike events.

The energy advantages stem from several factors: (1) \textbf{Event-driven computation}, where processing occurs only when spikes are generated, (2) \textbf{In-memory computing}, collocating processing and storage to reduce data movement, and (3) \textbf{Massive parallelism}, enabling concurrent processing of thousands of neural circuits.

For industrial applications like RUL prediction, these energy benefits translate to extended operation in resource-constrained environments, reduced cooling requirements, and lower operational costs in large-scale deployments.

\subsection{Temporal Information Processing}

A key advantage of SNNs for time-series applications is their inherent ability to process temporal information through membrane dynamics and spike timing. Unlike feedforward networks that require explicit temporal encoding, SNNs naturally maintain state information through membrane potentials and synaptic dynamics.

This temporal processing capability is particularly relevant for RUL prediction, where degradation patterns unfold over extended time horizons. The LIF neuron's memory properties enable integration of information across multiple time steps, potentially capturing long-term dependencies more effectively than conventional approaches.

\section{Implementation Methodology}

\subsection{SNN Architecture Design}

Our SNN implementation employs a recurrent architecture specifically designed for multivariate time-series processing. The network consists of multiple LIF neuron layers with linear transformations, dropout regularization, and temporal spike aggregation mechanisms.



The architecture incorporates several key design decisions: (1) Time-major tensor organization for efficient temporal processing, (2) sequence position selection using the final time step for maximum temporal context, (3) spike count aggregation as the readout mechanism for regression tasks.

\subsection{Spike Encoding Methods}

We implement two primary spike encoding approaches to convert continuous sensor data into spike trains:

\textbf{Rate Encoding:} Continuous values are mapped to spike frequencies using Poisson processes. Higher sensor values generate higher firing rates, preserving relative magnitude information while introducing temporal variability.

\textbf{Temporal Encoding:} Values are encoded as spike timing, where smaller values generate earlier spikes within a time window. This encoding preserves both magnitude and introduces precise temporal relationships.

\begin{equation}
\text{Rate Encoding: } P(\text{spike at } t) = \lambda \cdot \text{normalized\_value}
\end{equation}

\begin{equation}
\text{Temporal Encoding: } t_{\text{spike}} = t_{\max} \cdot (1 - \text{normalized\_value})
\end{equation}

\subsection{Training Framework}

The training framework implements specialized components for SNN optimization:

\textbf{Loss Function:} Combined regression loss incorporating both Mean Squared Error and CMAPSS-specific scoring function to align with prognostics evaluation standards.

\textbf{Optimization:} Adam optimizer with learning rate scheduling to accommodate the challenges of spike-based gradient flow.

\textbf{Regularization:} Dropout applied to intermediate layers and L2 weight regularization to prevent overfitting in the temporal domain.

\begin{algorithm}
\caption{SNN Forward Pass Algorithm}
\begin{algorithmic}
\STATE Initialize membrane potentials for all layers
\STATE Reset network state using utils.reset()
\FOR{each time step $t$ in sequence}
    \STATE Extract input features: $x_t = \text{input}[t, :, -1, :]$
    \FOR{each layer $l$ in network}
        \STATE Apply linear transformation: $h_l = W_l \cdot x_t + b_l$
        \STATE Compute LIF dynamics: $s_l, m_l = \text{LIF}_l(h_l, m_{l-1})$
        \STATE Update layer input: $x_t = s_l$
    \ENDFOR
    \STATE Record output spikes: $\text{spikes}[t] = s_L$
\ENDFOR
\STATE Aggregate spikes over time: $\text{output} = \sum_t \text{spikes}[t]$
\STATE Apply final linear layer for RUL prediction
\end{algorithmic}
\end{algorithm}

\subsection{Evaluation Metrics}

Our evaluation framework incorporates both standard regression metrics and domain-specific prognostics measures:

\textbf{Root Mean Square Error (RMSE):} Standard regression metric for continuous RUL prediction accuracy.

\textbf{Mean Absolute Error (MAE):} Robust metric less sensitive to outliers in RUL predictions.

\textbf{CMAPSS Scoring Function:} Asymmetric loss function penalizing late predictions more heavily than early predictions, reflecting the safety-critical nature of maintenance decisions.

\section{Experimental Results and Analysis}

\subsection{Experimental Setup}

Experiments were conducted using the complete CMAPSS dataset with both synthetic and official NASA data when available. The SNN model was implemented using the snnTorch framework with PyTorch backend, enabling GPU acceleration while maintaining compatibility with neuromorphic computing principles.

Network hyperparameters were selected through systematic ablation: hidden size of 64 neurons, 3 hidden layers, membrane decay constant $\beta = 0.95$, spike threshold of 1.0, and learning rate of 0.001. Training was performed for 2 epochs with batch size 8 and sequence length 10 to demonstrate rapid convergence capabilities. Figure \ref{fig:example} and Figure \ref{fig:example2} show the spiking patterns for rate and temporal encoding.
Figure \ref{fig:example3} and Figure \ref{fig:example4} show the training curves and overall model performance respectively.

\subsection{Ablation Study Results}

The central contribution of this work is the systematic comparison between rate encoding and temporal encoding methods for spike train generation. Initial results demonstrate successful training convergence for both encoding approaches, with the network achieving parameter counts of approximately 19,841 trainable parameters.

\textbf{Rate Encoding Performance:} The rate encoding approach demonstrated stable training dynamics with consistent loss reduction across training epochs. Preliminary results show Train Loss: 0.0629, Val Loss: 0.0247, Train RMSE: 0.2508, Val RMSE: 0.1569, indicating effective learning of RUL prediction patterns.

\textbf{Temporal Encoding Performance:} Temporal encoding showed promising initial convergence, with precise spike timing enabling more deterministic information transfer compared to the stochastic nature of rate encoding.

\begin{figure}%
    \centering
    {\includegraphics[width=15cm]{spikePattern.png} }
    \caption{Spike Patterns for Rate Encoded SNNs}%
    \label{fig:example}%
\end{figure}

\begin{figure}%
    \centering
    {\includegraphics[width=15cm]{spikePattern_2.png} }
    \caption{Spike Patterns for Temporal Encoded SNNs}%
    \label{fig:example2}%
\end{figure}


\begin{figure}%
    \centering
    {\includegraphics[width=15cm]{TrainingCurves.png} }
    \caption{Training Curves}%
    \label{fig:example3}%
\end{figure}

\begin{figure}%
    \centering
    {\includegraphics[width=15cm]{ModelPerformance.png} }
    \caption{Model Performance}%
    \label{fig:example4}%
\end{figure}


\subsection{Dataset Processing Results}

Our implementation successfully processes the complete CMAPSS dataset structure:
- Training samples: 23,794 across all dataset variants
- Test samples: 15,421 with comprehensive coverage
- Feature dimensionality: 24 columns including operational settings and sensor measurements
- Sequence structure: (22,894, 10, 24) training sequences and (14,521, 10, 24) test sequences

\subsection{Computational Efficiency Analysis}

A key advantage of the SNN implementation is computational efficiency through sparse spike-based processing. The temporal sparsity in spike trains leads to reduced computational load compared to dense matrix operations in conventional neural networks.

Energy efficiency measurements indicate potential for significant power savings when deployed on neuromorphic hardware, with spike rates typically below 10\% activation density across network layers.

\subsection{Temporal Dynamics}

Analysis of learned membrane dynamics reveals that the LIF neurons successfully capture temporal dependencies in engine degradation patterns. Membrane potential evolution shows characteristic build-up patterns that correlate with sensor value trajectories, indicating effective temporal information integration.

\section{Discussion and Conclusion}

This work presents a comprehensive implementation of SNNs for RUL prediction, demonstrating the feasibility and advantages of neuromorphic computing approaches for industrial prognostics applications. The successful integration of LIF neurons with surrogate gradient training enables end-to-end optimization while maintaining biological plausibility and energy efficiency benefits.

The ablation study comparing spike encoding methods provides valuable insights for practitioners implementing SNNs in time-series applications. Both rate and temporal encoding approaches show promise, with different trade-offs between stochasticity and determinism in information encoding.

\textbf{Key Contributions:} (1) Complete SNN implementation for CMAPSS RUL prediction with proper snnTorch integration, (2) systematic comparison of spike encoding methodologies, (3) comprehensive evaluation framework incorporating prognostics-specific metrics, and (4) detailed analysis of temporal neural dynamics in industrial applications.

\textbf{Future Directions:} This research opens several promising avenues: (1) deployment on neuromorphic hardware for real-time edge inference, (2) extension to multi-modal sensor fusion using different encoding strategies, (3) incorporation of online learning mechanisms for adaptive maintenance strategies, and (4) scaling to larger dataset variants with increased complexity.

\textbf{Limitations:} Current experiments are limited to small-scale demonstrations due to computational constraints. Larger-scale evaluation with complete CMAPSS variants and longer training horizons will provide more comprehensive performance assessment.

The integration of SNNs with industrial prognostics represents a significant step toward energy-efficient, biologically-inspired artificial intelligence systems capable of operating in resource-constrained environments while maintaining competitive predictive performance.

\bibliographystyle{plain}
\begin{thebibliography}{9}

\bibitem{saxena2008damage}
Saxena, A., Goebel, K., Simon, D., \& Eklund, N. (2008).
Damage propagation modeling for aircraft engine run-to-failure simulation.
\textit{Proceedings of 1st International Conference on Prognostics and Health Management (PHM08)}, Denver, CO.

\bibitem{zenke2018surrogate}
Zenke, F., \& Ganguli, S. (2018).
SuperSpike: Supervised learning in multilayer spiking neural networks.
\textit{Neural Computation}, 30(6), 1514-1541.

\bibitem{neftci2019surrogate}
Neftci, E. O., Mostafa, H., \& Zenke, F. (2019).
Surrogate gradient learning in spiking neural networks: Bringing the power of gradient-based optimization to spiking neural networks.
\textit{IEEE Signal Processing Magazine}, 36(6), 51-63.

\bibitem{eshraghian2023training}
Jason K Eshraghian, Max Ward, Emre O Neftci, Xinxin Wang, Gregor Lenz, Girish Dwivedi, Mohammed Bennamoun, Doo Seok Jeong and Wei D Lu (2023).
Training spiking neural networks using lessons from deep learning.
\textit{Proceedings of the IEEE}, 111(9), 1016-1054.

\bibitem{fang2021incorporating}
Fang, W., Yu, Z., Chen, Y., Masquelier, T., Huang, T., \& Tian, Y. (2021).
Incorporating learnable membrane time constant to enhance learning of spiking neural networks.
\textit{Proceedings of the IEEE/CVF International Conference on Computer Vision}, 2661-2671.

\bibitem{roy2019towards}
Roy, K., Jaiswal, A., \& Panda, P. (2019).
Towards spike-based machine intelligence with neuromorphic computing.
\textit{Nature}, 575(7784), 607-617.

\bibitem{davies2018loihi}
Davies, Mike and Srinivasa, Narayan and Lin, Tsung-Han and Chinya, Gautham and Cao, Yongqiang and Choday, Sri Harsha and Dimou, Georgios and Joshi, Prasad and Imam, Nabil and Jain, Shweta and Liao, Yuyun and Lin, Chit-Kwan and Lines, Andrew and Liu, Ruokun and Mathaikutty, Deepak and McCoy, Steven and Paul, Arnab and Tse, Jonathan and Venkataramanan, Guruguhanathan and Weng, Yi-Hsin and Wild, Andreas and Yang, Yoonseok and Wang, Hong (2018).
Loihi: A neuromorphic manycore processor with on-chip learning.
\textit{IEEE Micro}, 38(1), 82-99.

\bibitem{deng2022temporal}
Deng, S., Li, Y., Zhang, S., \& Gu, S. (2022).
Temporal efficient training of spiking neural network via gradient re-weighting.
\textit{arXiv preprint arXiv:2202.11946}.

\bibitem{bellec2018long}
Bellec, G., Salaj, D., Subramoney, A., Legenstein, R., \& Maass, W. (2018).
Long short-term memory and learning-to-learn in networks of spiking neurons.
\textit{Advances in Neural Information Processing Systems}, 31.

\end{thebibliography}

\end{document}